\section{Einf�hrung}

\mode<presentation>{
  \begin{frame}<presentation> \frametitle{Inhalt}
    \tableofcontents[currentsection,sectionstyle=show/hide,subsectionstyle=show/show/hide]
  \end{frame}
}

\begin{frame}
\frametitle{Was ist HDNUM}
\begin{itemize}
\item HDNUM ist eine kleine Sammlung von C++ Klassen, die die
  Implementierung numerischer Algorithmen aus der Vorlesung
  erleichtern soll.

\item Die aktuelle Version gibt es unter
\end{itemize}

\begin{center}
  {\small\url{http://conan.iwr.uni-heidelberg.de/teaching/numerik1_ws2011/}}
\end{center}

\begin{itemize}
\item Einige Ziele bei der Entwicklung von HDNUM waren:
\begin{itemize}
\item Einfache Installation: Es mur nur eine Header-Datei eingebunden werden.
\item Einfache Benutzung der Klassen: Z.B. keine dynamische
  Speicherverwaltung.
\item M�glichkeit der Rechnung mit verschiedenen Zahl-Datentypen.
\item Effiziente Realisierung der Verfahren m�glich:
  Z.B. Block-Algorithmen in der linearen Algebra.
\end{itemize}
\end{itemize}

\end{frame}

\begin{frame}
\frametitle{Installation}
\begin{itemize}
\item Datei \lstinline{hdnum-x.yy.tgz} (komprimiertes tar archive)
  herunterladen.
\item Archiv mit \lstinline{tar zxf hdnum-x.yy.tgz} entpacken.
\item Das Verzeichnis enth�lt unter anderem:
\begin{itemize}
\item Das Verzeichnis \lstinline{src} mit dem Quellcode der Klassen
  (muss Sie nicht interessieren).
\item Das Verzeichnis \lstinline{examples} mit den Beispielanwendungen
  (die sollten Sie sich ansehen).
\item Das Verzeichnis \lstinline{tutorial}: Quelle f�r dieses Dokument.
\item Die Datei \lstinline{hdnum.hh}, die zentrale Header-Datei, die
  in alle Anwendungen eingebunden werden muss.
\end{itemize}
\item Das Verzeichnis \lstinline{hdnum/examples/num0} enth�lt ein simples
  Makefile zum �bersetzen der Programme.
\item Die Beispiele erfordern die Installation der GNU
  multiprecision library \url{http://gmplib.org/}. Ist diese nicht
  vorhanden m�ssen Makefiles entsprechend angepasst werden.
\end{itemize}

\end{frame}

\begin{frame}
\frametitle{Typisches HDNUM Programm}

\lstinputlisting[basicstyle=\ttfamily\scriptsize,numbers=left,
numberstyle=\tiny, numbersep=5pt]{../examples/num0/hallohdnum.cc}

\begin{itemize}
\item �bersetzen im Verzeichnis \lstinline{examples/num0} mit GMP installiert:

{\footnotesize\lstinline{g++ -I.. -o hallohdnum hallohdnum.cc -lm -lgmpxx -lgmp}}
\item und ohne GMP:

{\footnotesize\lstinline{g++ -I.. -o hallohdnum hallohdnum.cc -lm}}
\item oder einfach

{\footnotesize\lstinline{make}}
\item oder falls kein GMP installiert ist

{\footnotesize\lstinline{make nogmp}}
\end{itemize}
\end{frame}
