\section{Gew�hnliche Differentialgleichungen}

\mode<presentation>{
  \begin{frame}<presentation> \frametitle{Inhalt}
    \tableofcontents[currentsection,sectionstyle=show/hide,subsectionstyle=show/show/hide]
  \end{frame}
}

\subsection{Differentialgleichungsmodelle und L�ser}

\begin{frame}[fragile]
\frametitle{Gew�hnliche Differentialgleichungen in HDNUM}
\begin{itemize}
\item Erlaube L�sung beliebiger Modelle mit beliebigen L�sern.
\item Erlaube variable Typen f�r Zeit und Zustand.
\item Trenne folgende Komponenten:
\begin{itemize}
\item Differentialgleichungsmodell (inklusive Anfangsbedingung),
\item L�sungsverfahren,
\item Steuerung und Zeitschleife.
\end{itemize}
\end{itemize}
\end{frame}

\begin{frame}[fragile]
\frametitle{Differentialgleichungsmodell}
Ein Differentialgleichungsmodell ist gegeben durch
\begin{itemize}
\item Typen f�r Zeit und Zustandskomponenten variabel.
\item Gr��e des Systems $d$.
\item Anfangszustand $(t_0,u_0)$.
\item Funktion $f(t,x) : \mathbb{R}\times\mathbb{R}^d \to \mathbb{R}^d$.
\item Optional die Jacobimatrix $f_x(t,x)$ (wird f�r implizite Verfahren ben�tigt).
\item F�r Zustand und Jacobimatrix verwenden wir Vektor- und Matrixklassen aus HDNUM.
\end{itemize}
Als n�chstes ein Beispiel f�r das Modellproblem
\begin{equation*}
u'(t) = \lambda u(t), \quad t\geq t_0, \quad u(t_0)=u_0, \quad \lambda\in\mathbb{R}, \mathbb{C}.
\end{equation*}
\end{frame}

\begin{frame}[fragile,allowframebreaks,allowdisplaybreaks]
\frametitle{Modellproblem}
\framesubtitle{(Datei \texttt{examples/num1/modelproblem.hh})}
\lstinputlisting[basicstyle=\tiny,numbers=left,
numberstyle=\tiny, numbersep=2pt]{../examples/num1/modelproblem.hh}
\end{frame}

\begin{frame}[fragile]
\frametitle{Differentialgleichungsl�ser}
\begin{itemize}
\item Differentialgleichungsmodell ist ein Template-Parameter.
\item Typen f�r Zeit und Zustand werden aus Differentialgleichungsmodell genommen.
\item Kapselt aktuellen Zustand und aktuelle Zeit (und evtl. weitere Zust�nde).
\item Methode \lstinline{step} f�hrt einen Schritt des Verfahrens durch.
\end{itemize}
Als n�chstes ein Beispiel f�r den expliziten Euler.
\end{frame}

\begin{frame}[fragile,allowframebreaks,allowdisplaybreaks]
\frametitle{Expliziter Euler}
\framesubtitle{(Datei \texttt{examples/num1/expliciteuler.hh})}
\lstinputlisting[basicstyle=\tiny,numbers=left,
numberstyle=\tiny, numbersep=2pt]{../examples/num1/expliciteuler.hh}
\end{frame}

\begin{frame}[fragile]
\frametitle{L�sung und Ergebnisausgabe}
Die L�sung eines Differentialgleichungsmodells besteht nun aus
\begin{itemize}
\item Instantieren der entsprechenden Objekte f�r Modell und L�ser.
\item Zeitschrittschleife bis zur gew�nschten Endzeit.
\item Speicherung und Ausgabe der Ergebnisse in einem \lstinline{hdnum::Vector}.
\item Visualisierung der Ergebnisse mit \lstinline{gnuplot}.
\end{itemize}
\end{frame}

\begin{frame}[fragile,allowframebreaks,allowdisplaybreaks]
\frametitle{Hauptprogramm f�r Modellproblem}
\framesubtitle{(Datei \texttt{examples/num1/modelproblem.cc})}
\lstinputlisting[basicstyle=\tiny,numbers=left,
numberstyle=\tiny, numbersep=2pt]{../examples/num1/modelproblem.cc}
\end{frame}
