\documentclass[a4paper,11pt]{article}
\usepackage[ngerman]{babel}
\usepackage[utf8]{inputenc}
\usepackage[T1]{fontenc}
\usepackage[a4paper,total={150mm,240mm}]{geometry}
\usepackage{amsmath}
\usepackage{amsfonts}
\usepackage{amsthm}
\usepackage{amscd}
\usepackage{grffile}
\usepackage{tikz} 
\usetikzlibrary{3d,calc}
\usepackage{eurosym} 
\usepackage{graphicx}
\usepackage{color}
\usepackage{listings}
\lstset{language=C++, basicstyle=\ttfamily, 
  keywordstyle=\color{black}\bfseries, tabsize=4, stringstyle=\ttfamily,
  commentstyle=\itshape, extendedchars=true, escapeinside={/*@}{@*/}}
\usepackage{paralist}
\usepackage{curves}
\usepackage{calc}
\usepackage{picinpar}
\usepackage{enumerate}
\usepackage{algpseudocode}
\usepackage{bm}
\usepackage{multibib}
\usepackage{hyperref}
\usepackage{textcase}
\usepackage{nicefrac}
\usepackage[titletoc,toc,title]{appendix}

\definecolor{listingbg}{gray}{0.95}

\theoremstyle{definition}
\newtheorem{exm}{Example}
\newtheorem{thm}{Theorem}
\newtheorem{cor}{Corollary}
\newtheorem{obs}{Observation}

\title{HDNUM\\ Heidelberger Numerikbibliothek}
\author{Peter Bastian\\
  Universität Heidelberg, \\
  Interdisziplinäres Zentrum für Wissenschaftliches Rechnen\\
  Im Neuenheimer Feld 368, D-69120 Heidelberg\\
  \url{Peter.Bastian@iwr.uni-heidelberg.de}
}
\date{\today}

\begin{document}

\maketitle

\begin{abstract}
Die Heidelberger Numerikbibliothek wurde begleitend zu den Vorlesungen
\textit{Einführung in die Numerik} und \textit{Numerik} in der Programmiersprache C++
entwickelt und stellt
einfach zu benutzende Klassen für grundlegende Aufgaben in der Numerik bis hin zur Lösung
von gewöhnlichen Differentialgleichungen zur Verfügung. In fast allen Klassen ist der
benutzte Zahlentyp parametrisierbar so dass auch hochpräzise Rechnungen durchgeführt werden können.
\end{abstract}

\section{Einführung}

\subsection{Was ist HDNUM}

Die Heidelberger Numerikbibliothek (HDNUM) ist eine C++-basierte Bibliothek
zur Durchführung der praktischen Übungen zu den Vorlesungen \textit{Einführung in die
Numerik} und \textit{Numerik (gewöhnlicher Differentialgleichungen)}. Die aktuelle Version ist
unter
\begin{center}
\url{https://conan2.iwr.uni-heidelberg.de/git/Teaching/hdnum}
\end{center}
verfügbar und wird mit dem Versionskontrollsystem \lstinline{git} verwaltet.
Spezifische Versionen können auf der jeweiligen Vorlesungswebseite veröffentlicht werden.

Ziele bei der Entwicklung von HDNUM waren i) die einfache Benutzbarkeit (inklusive einfacher 
Installation), 
ii) die Demonstration objektorientierter Programmierung in der Numerik sowie die
Möglichkeit zur Durchführung von Berechnungen mit beliebiger Genauigkeit auf Basis
der Gnu Multiple Precision\footnote{\url{https://gmplib.org}} Bibliothek.
Derzeit bietet HDNUM die folgende Funktionalität:
\begin{enumerate}[1)]
\item Klassen für Matrizen und Vektoren
\item Lösung linearer Gleichungssystem
\item Lösung nichtlinearer Gleichungssysteme
\item Lösung von Systemen gewöhnlicher Differentialgleichungen
\item Lösung der Poissongleichung mit Finiten Differenzen
\end{enumerate}

\subsection{Installation}

\subsection{So legt man los}

\section{Lineare Algebra}

\subsection{Vektoren}

\subsection{Matrizen}

\subsection{LR Zerlegung}

\subsection{QR Zerlegung}

\begin{appendices}

\section{Kleiner Programmierkurs}

\section{Unix Kommandos}

\end{appendices}

\bibliographystyle{plain}
\bibliography{hdnum.bib}
 
\end{document}
