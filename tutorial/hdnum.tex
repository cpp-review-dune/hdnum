\documentclass[a4paper,11pt]{article}
\usepackage[ngerman]{babel}
\usepackage[utf8]{inputenc}
\usepackage[T1]{fontenc}
\usepackage[a4paper,total={150mm,240mm}]{geometry}
\usepackage{amsmath}
\usepackage{amsfonts}
\usepackage{amsthm}
\usepackage{amscd}

\usepackage[envcountsect,noamsthm]{beamerarticle}

\usepackage{grffile}
\usepackage{tikz} 
\usetikzlibrary{3d,calc}
\usepackage{eurosym} 
\usepackage{graphicx}
\usepackage{color}
\definecolor{listingbg}{gray}{0.95}
\usepackage{listings}
\lstset{language=C++, basicstyle=\ttfamily, 
  keywordstyle=\color{black}\bfseries, tabsize=4, stringstyle=\ttfamily,
  commentstyle=\itshape, extendedchars=false,backgroundcolor=\color{listingbg},escapeinside={/*@}{@*/}}
\usepackage{paralist}
\usepackage{curves}
\usepackage{calc}
\usepackage{picinpar}
\usepackage{enumerate}
\usepackage{algpseudocode}
\usepackage{bm}
\usepackage{multibib}
\usepackage{hyperref}
\usepackage{textcase}
\usepackage{nicefrac}
\usepackage[titletoc,toc,title]{appendix}

\theoremstyle{definition}
\newtheorem{exm}{Example}
\newtheorem{thm}{Theorem}
\newtheorem{cor}{Corollary}
\newtheorem{obs}{Observation}

\title{HDNUM\\ Heidelberger Numerikbibliothek}
\author{Peter Bastian\\
  Universität Heidelberg, \\
  Interdisziplinäres Zentrum für Wissenschaftliches Rechnen\\
  Im Neuenheimer Feld 368, D-69120 Heidelberg\\
  \url{Peter.Bastian@iwr.uni-heidelberg.de}
}
\date{\today}

\begin{document}

\maketitle

\begin{abstract}
Die Heidelberger Numerikbibliothek wurde begleitend zu den Vorlesungen
\textit{Einführung in die Numerik} und \textit{Numerik} in der Programmiersprache C++
entwickelt und stellt
einfach zu benutzende Klassen für grundlegende Aufgaben in der Numerik bis hin zur Lösung
von gewöhnlichen Differentialgleichungen zur Verfügung. In fast allen Klassen ist der
benutzte Zahlentyp parametrisierbar so dass auch hochpräzise Rechnungen durchgeführt werden können.
\end{abstract}

\section{Einführung}

\subsection{Was ist HDNUM}

Die Heidelberger Numerikbibliothek (HDNUM) ist eine C++-basierte Bibliothek
zur Durchführung der praktischen Übungen zu den Vorlesungen \textit{Einführung in die
Numerik} und \textit{Numerik (gewöhnlicher Differentialgleichungen)}. Die aktuelle Version ist
unter
\begin{center}
\url{https://parcomp-git.iwr.uni-heidelberg.de/Teaching/hdnum}
\end{center}
verfügbar und wird mit dem Versionskontrollsystem \lstinline{git} verwaltet.
Spezifische Versionen können auf der jeweiligen Vorlesungswebseite veröffentlicht werden.

Ziele bei der Entwicklung von HDNUM waren i) die einfache Benutzbarkeit (inklusive einfacher 
Installation), 
ii) die Demonstration objektorientierter Programmierung in der Numerik sowie die
Möglichkeit zur Durchführung von Berechnungen mit beliebiger Genauigkeit auf Basis
der Gnu Multiple Precision\footnote{\url{https://gmplib.org}} Bibliothek.
Derzeit bietet HDNUM die folgende Funktionalität:
\begin{enumerate}[1)]
\item Klassen für Matrizen und Vektoren
\item Lösung linearer Gleichungssystem
\item Lösung nichtlinearer Gleichungssysteme
\item Lösung von Systemen gewöhnlicher Differentialgleichungen
\item Lösung der Poissongleichung mit Finiten Differenzen
\end{enumerate}

\subsection{Installation}

HDNUM ist eine \glqq{}header only\grqq{} Bibliothek und erfordert keine Installation
ausser dem Herunterladen der Dateien. Die aktuelle Version kann man mittels
\begin{lstlisting}[basicstyle=\ttfamily\footnotesize,language=bash,frame=single]
$ git clone https://parcomp-git.iwr.uni-heidelberg.de/Teaching/hdnum.git
\end{lstlisting}
herunterladen. Hierzu ist das Programm \lstinline{git} erforderlich, welches für alle
Betriebssysteme frei erhältlich ist. Alternativ wird üblicherweise auch ein 
komprimiertes \lstinline{tar}-archive auf der Homepage der jeweiligen Vorlesung angeboten.
Dies entpackt man mittels
\begin{lstlisting}[basicstyle=\ttfamily\small,frame=single]
$ tar zxvf hdnum-XX.tgz
\end{lstlisting}
In dem entpackten bzw. installiertem Verzeichnis findet man die folgenden Dateien und
Unterverzeichnisse:
\begin{itemize}
\item \lstinline{hdnum.hh}: Diese Header-Datei ist in ein C++-Programm einzubinden um HDNUM
nutzen zu können. 
\item Das Verzeichnis \lstinline{mystuff} ist für ihre Programme vorgesehen aber Sie können
natürlich jedes andere Verzeichnis nutzen. Wichtig ist nur, dass der Compiler die Datei 
\lstinline{hdnum.hh} findet. Im Verzeichnis \lstinline{mystuff} ist schon ein Beispielprogramm
um gleich loslegen zu können. Dieses Programm übersetzt man mit:
\begin{lstlisting}[basicstyle=\ttfamily\small,frame=single]
$ cd mystuff
$ g++ -I.. -o example example.cc
\end{lstlisting}
Diese Befehle setzen voraus, dass auf ihrem System der GNU C++-Compiler installiert ist.
Unter Windows oder für andere Compiler müssen Sie die Befehle entsprechend anpassen.
\item Das Verzeichnis \lstinline{examples} im HDNUM-Ordner enthält viele Beispiele
geordnet nach Programmierkurs, Numerik 0 und Numerik 1.
\item Das Verzeichnis \lstinline{src} im HDNUM-Ordner enthält den Quellcode
der HDNUM Bibliothek. Diese Dateien werden von \lstinline{hdnum.hh} eingebunden.
\item Das Verzeichnis \lstinline{programmierkurs} im HDNUM-Ordner enthält
die Folien zum Programmierkurs.
\item Das Verzeichnis \lstinline{tutorial} im HDNUM-Ordner enthält den Quellcode
für dieses Dokument.
\end{itemize}

\subsubsection*{GNU Multiple Precision Bibliothek}

HDNUM kann Berechnungen mit hoher Genauigkeit durchführen. Hierzu ist die GNU Multiple Precision
Bibliothek (GMP) erforderlich, 
welche Sie für viele Systeme kostenlos erhalten können. Um GMP nutzen zu können müssen 
Sie in der Datei \lstinline{hdnum.hh} die Zeile
\begin{lstlisting}[basicstyle=\ttfamily\small,frame=single]
#define HDNUM_HAS_GMP 1
\end{lstlisting}
\textit{auskommentieren}. Zusätzlich sind eventuell Compileroptionen notwendig damit der Compiler
die Headerdateien und Bibliotheken von GMP findet. Dies kann dann so aussehen:
\begin{lstlisting}[basicstyle=\ttfamily\footnotesize,frame=single]
$ g++ -I.. -I/opt/local/include -o example example.cc -L/opt/local/lib -lgmpxx -lgmp
\end{lstlisting}

\section{Lineare Algebra}

\subsection{Vektoren}

\begin{frame}[fragile]
\frametitle{\lstinline{hdnum::Vector<T>}}
\begin{itemize}
\item \lstinline{hdnum::Vector<T>} ist ein Klassen-Template.
\item Es macht aus einem beliebigen (Zahl-)Datentypen \lstinline{T}
  einen Vektor.
\item Auch komplexe und hochgenaue Zahlen sind möglich.
\item Vektoren verhalten sich so wie man es aus der Mathematik kennt:
\begin{itemize}
\item Bestehen aus $n$ Komponenten.
\item Diese sind von $0$ bis $n-1$ (!) durchnummeriert.
\item Addition und Multiplikation mit Skalar.
\item Skalarprodukt und Euklidsche Norm
\item Matrix-Vektor-Multiplikation
\end{itemize}
\item Die folgenden Beispiele findet man in \lstinline{vektoren.cc}
\end{itemize}
\end{frame}

\begin{frame}[fragile]
\frametitle{Konstruktion und Zugriff}
\begin{itemize}
\item Konstruktion mit und ohne Initialisierung\\
{\footnotesize{\begin{lstlisting}{}
hdnum::Vector<float> x(10);        // Vektor mit 10 Elementen
hdnum::Vector<double> y(10,3.14);  // 10 Elemente initialisiert
hdnum::Vector<float> a;            // ein leerer Vektor
\end{lstlisting}}}
\item Speziellere Vektoren\\
{\footnotesize{\begin{lstlisting}{}
hdnum::Vector<std::complex<double> >
  cx(7,std::complex<double>(1.0,3.0));
mpf_set_default_prec(1024); // Setze Genauigkeit fuer mpf_class
hdnum::Vector<mpf_class> mx(7,mpf_class("4.44"));
\end{lstlisting}}}
\item Zugriff auf Element\\
{\footnotesize{\begin{lstlisting}{}
for (std::size_t i=0; i<x.size(); i=i+1)
  x[i] = i;                 // Zugriff auf Elemente
\end{lstlisting}}}
\item Vektorobjekt wird am Ende des umgebenden Blockes gelöscht.
\end{itemize}
\end{frame}

\begin{frame}[fragile]
\frametitle{Kopie und Zuweisung}
\begin{itemize}
\item Copy-Konstruktor: Erstellen eines Vektors als Kopie eines anderen
{\footnotesize{\begin{lstlisting}{}
hdnum::Vector<float> z(x); // z ist Kopie von x
\end{lstlisting}}}
\item Zuweisung, auch die Größe wird überschrieben!
{\footnotesize{\begin{lstlisting}{}
b = z;              // b kopiert die Daten aus z
a = 5.4;            // Zuweisung an alle Elemente
hdnum::Vector<double> w;   // leerer Vektor
w.resize(x.size()); // make correct size
w = x;              // copy elements
\end{lstlisting}}}
\item Ausschnitte von Vektoren\\
{\footnotesize{\begin{lstlisting}{}
hdnum::Vector<float> w(x.sub(7,3));// w ist Kopie von x[7],...,x[9]
z = x.sub(3,4);             // z ist Kopie von x[3],...,x[6]
\end{lstlisting}}}
\end{itemize}
\end{frame}

\begin{frame}[fragile]
\frametitle{Rechnen und Ausgabe}
\begin{itemize}
\item Vektorraumoperationen und Skalarprodukt\\
{\footnotesize{\begin{lstlisting}{}
w += z;            // w = w+z
w -= z;            // w = w-z
w *= 1.23;         // skalare Multiplikation
w /= 1.23;         // skalare Division
w.update(1.23,z);  // w = w + a*z
float s;
s = w*z;           // Skalarprodukt
\end{lstlisting}}}
\item Ausgabe auf die Konsole\\
{\footnotesize{\begin{lstlisting}{}
std::cout << w << std::endl;// schoene Ausgabe
w.iwidth(2);                // Stellen in Indexausgabe
w.width(20);                // Anzahl Stellen gesamt
w.precision(16);            // Anzahl Nachkommastellen
std::cout << w << std::endl;// nun mit mehr Stellen
std::cout <<cx << std::endl;// geht auch fuer complex
std::cout <<mx << std::endl;// geht auch fuer mpf_class
\end{lstlisting}}}
\end{itemize}
\end{frame}

\begin{frame}[fragile]
\frametitle{Beispielausgabe}
{\footnotesize{\begin{lstlisting}{}
[   0]    1.204200e+01
[   1]    1.204200e+01
[   2]    1.204200e+01
[   3]    1.204200e+01

[ 0] 1.2042000770568848e+01
[ 1] 1.2042000770568848e+01
[ 2] 1.2042000770568848e+01
[ 3] 1.2042000770568848e+01
\end{lstlisting}}}
\end{frame}

\begin{frame}[fragile]
\frametitle{Hilfsfunktionen}
{\footnotesize{\begin{lstlisting}{}
float d = norm(w);          // Euklidsche Norm
d = w.two_norm();           // das selbe
zero(w);                    // das selbe wie w=0.0
fill(w,(float)1.0);         // das selbe wie w=1.0
fill(w,(float)0.0,(float)0.1); // w[0]=0, w[1]=0.1, w[2]=0.2, ...
unitvector(w,2);            // kartesischer Einheitsvektor
gnuplot("test.dat",w);      // gnuplot Ausgabe: i w[i]
gnuplot("test2.dat",w,z);   // gnuplot Ausgabe: w[i] z[i]
\end{lstlisting}}}
\end{frame}

\begin{frame}[fragile]
\frametitle{Funktionen}
\begin{itemize}
\item Beispiel: Summe aller Komponenten\\
{\footnotesize{\begin{lstlisting}{}
double sum (hdnum::Vector<double> x) {
  double s(0.0);
  for (std::size_t i=0; i<x.size(); i=i+1)
    s = s + x[i];
  return s;
}
\end{lstlisting}}}
\item Verbesserte Version mit \textbf{Funktionentemplate} und by-const-reference Übergabge\\
{\footnotesize{\begin{lstlisting}{}
template<class T>
T sum (const hdnum::Vector<T>& x) {
  T s(0.0);
  for (std::size_t i=0; i<x.size(); i=i+1)
    s = s + x[i];
  return s;
}
\end{lstlisting}}}
\item By-value Übergabe ist be großen Objekten vorzuziehen
\end{itemize}
\end{frame}

\subsection{Matrizen}

\begin{frame}[fragile]
\frametitle{\lstinline{hdnum::DenseMatrix<T>}}
\begin{itemize}
\item \lstinline{hdnum::DenseMatrix<T>} ist ein Klassen-Template.
\item Es macht aus einem beliebigen (Zahl-)Datentypen \lstinline{T}
  eine Matrix.
\item Auch komplexe und hochgenaue Zahlen sind möglich.
\item Matrizen verhalten sich so wie man es aus der Mathematik kennt:
\begin{itemize}
\item Bestehen aus $m\times n$ Komponenten.
\item Diese sind von $0$ bis $m-1$ bzw. $n-1$ (!) durchnummeriert.
\item $m\times n$-Matrizen bilden einen Vektorraum.
\item Matrix-Vektor und Matrizenmultiplikation.
\end{itemize}
\item Die folgenden Beispiele findet man in \lstinline{matrizen.cc}
\end{itemize}
\end{frame}

\begin{frame}[fragile]
\frametitle{Konstruktion und Zugriff}
\begin{itemize}
\item Konstruktion mit und ohne Initialisierung\\
{\footnotesize{\begin{lstlisting}{}
hdnum::DenseMatrix<float> B(10,10);     // 10x10 Matrix uninitialisiert
hdnum::DenseMatrix<float> C(10,10,0.0); // 10x10 Matrix initialisiert
\end{lstlisting}}}
\item Zugriff auf Elemente\\
{\footnotesize{\begin{lstlisting}{}
for (int i=0; i<B.rowsize(); ++i)
  for (int j=0; j<B.colsize(); ++j)
    B[i][j] = 0.0;          // jetzt ist B initialisiert
\end{lstlisting}}}
\item Matrixobjekt wird am Ende des umgebenden Blockes gelöscht.
\end{itemize}
\end{frame}

\begin{frame}[fragile]
\frametitle{Kopie und Zuweisung}
\begin{itemize}
\item Copy-Konstruktor: Erstellen einer Matrix als Kopie einer anderen
{\footnotesize{\begin{lstlisting}{}
hdnum::DenseMatrix<float> D(B); // D Kopie von B
\end{lstlisting}}}
\item Zuweisung, kopiert auch Größe mit
{\footnotesize{\begin{lstlisting}{}
hdnum::DenseMatrix<float> A(B.rowsize(),B.colsize()); // korrekte Groesse
A = B;                    // kopieren
\end{lstlisting}}}
\item Ausschnitte von Matrizen (Untermatrizen)\\
{\footnotesize{\begin{lstlisting}{}
hdnum::DenseMatrix<float> F(A.sub(1,2,3,4));// 3x4 Mat ab (1,2)
\end{lstlisting}}}
\end{itemize}
\end{frame}

\begin{frame}[fragile]
\frametitle{Rechnen mit Matrizen}
\begin{itemize}
\item Vektorraumoperationen (Vorsicht: Matrizen sollten passende Größe haben!)\\
{\footnotesize{\begin{lstlisting}{}
A += B;           // A = A+B
A -= B;           // A = A-B
A *= 1.23;        // Multiplikation mit Skalar
A /= 1.23;        // Division durch Skalar
A.update(1.23,B); // A = A + s*B
\end{lstlisting}}}
\item Matrix-Vektor und Matrizenmultiplikation\\
{\footnotesize{\begin{lstlisting}{}
hdnum::Vector<float> x(10,1.0); // make two vectors
hdnum::Vector<float> y(10,2.0);
A.mv(y,x);               // y = A*x
A.umv(y,x);              // y = y + A*x
A.umv(y,(float)-1.0,x);  // y = y + s*A*x
C.mm(A,B);               // C = A*B
C.umm(A,B);              // C = C + A*B
A.sc(x,1);               // mache x zur ersten Spalte von A
A.sr(x,1);               // mache x zur ersten Zeile von A
float d=A.norm_infty();  // Zeilensummennorm
d=A.norm_1();            // Spaltensummennorm
\end{lstlisting}}}
\end{itemize}
\end{frame}

\begin{frame}[fragile]
\frametitle{Ausgabe und Hilfsfunktionen}
\begin{itemize}
\item Ausgabe von Matrizen\\
{\footnotesize{\begin{lstlisting}{}
std::cout << A.sub(0,0,3,3) << std::endl;// schöne Ausgabe
A.iwidth(2);                // Stellen in Indexausgabe
A.width(10);                // Anzahl Stellen gesamt
A.precision(4);             // Anzahl Nachkommastellen
std::cout << A << std::endl;// nun mit mehr Stellen
\end{lstlisting}}}
\item einige Hilfsfunktionen
{\footnotesize{\begin{lstlisting}{}
identity(A);
spd(A);
fill(x,(float)1,(float)1);
vandermonde(A,x);
\end{lstlisting}}}
\end{itemize}
\end{frame}

\begin{frame}[fragile]
\frametitle{Beispielausgabe}
{\footnotesize{\begin{lstlisting}{}
               0           1           2           3
  0   4.0000e+00 -1.0000e+00 -2.5000e-01 -1.1111e-01
  1  -1.0000e+00  4.0000e+00 -1.0000e+00 -2.5000e-01
  2  -2.5000e-01 -1.0000e+00  4.0000e+00 -1.0000e+00
  3  -1.1111e-01 -2.5000e-01 -1.0000e+00  4.0000e+00
\end{lstlisting}}}
\end{frame}

\begin{frame}[fragile]
\frametitle{Funktion mit Matrixargument}
Beispiel einer Funktion, die eine Matrix $A$ und einen Vektor $b$
initialisiert.

{\footnotesize{\begin{lstlisting}{}
template<class T>
void initialize (hdnum::DenseMatrix<T>& A, hdnum::Vector<T>& b)
{
  if (A.rowsize()!=A.colsize() || A.rowsize()==0)
    HDNUM_ERROR("need square and nonempty matrix");
  if (A.rowsize()!=b.size())
    HDNUM_ERROR("b must have same size as A");
  for (int i=0; i<A.rowsize(); ++i)
    {
      b[i] = 1.0;
      for (int j=0; j<A.colsize(); ++j)
        if (j<=i) A[i][j]=1.0; else A[i][j]=0.0;
    }
}
\end{lstlisting}}}
\end{frame}

\subsection{LR Zerlegung}

\subsection{QR Zerlegung}

\begin{appendices}

\section{Kleiner Programmierkurs}

\section{Unix Kommandos}

\end{appendices}

\bibliographystyle{plain}
\bibliography{hdnum.bib}
 
\end{document}
