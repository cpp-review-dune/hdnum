\documentclass[a4paper,11pt]{article}
\usepackage[ngerman]{babel}
\usepackage[utf8]{inputenc}
\usepackage[T1]{fontenc}
\usepackage[a4paper,total={150mm,240mm}]{geometry}
\usepackage{amsmath}
\usepackage{amsfonts}
\usepackage{amsthm}
\usepackage{amscd}
\usepackage{grffile}
\usepackage{tikz} 
\usetikzlibrary{3d,calc}
\usepackage{eurosym} 
\usepackage{graphicx}
\usepackage{color}
\definecolor{listingbg}{gray}{0.95}
\usepackage{listings}
\lstset{language=C++, basicstyle=\ttfamily, 
  keywordstyle=\color{black}\bfseries, tabsize=4, stringstyle=\ttfamily,
  commentstyle=\itshape, extendedchars=false,backgroundcolor=\color{listingbg},escapeinside={/*@}{@*/}}
\usepackage{paralist}
\usepackage{curves}
\usepackage{calc}
\usepackage{picinpar}
\usepackage{enumerate}
\usepackage{algpseudocode}
\usepackage{bm}
\usepackage{multibib}
\usepackage{hyperref}
\usepackage{textcase}
\usepackage{nicefrac}
\usepackage[titletoc,toc,title]{appendix}

\theoremstyle{definition}
\newtheorem{exm}{Example}
\newtheorem{thm}{Theorem}
\newtheorem{cor}{Corollary}
\newtheorem{obs}{Observation}

\title{HDNUM\\ Heidelberger Numerikbibliothek}
\author{Peter Bastian\\
  Universität Heidelberg, \\
  Interdisziplinäres Zentrum für Wissenschaftliches Rechnen\\
  Im Neuenheimer Feld 368, D-69120 Heidelberg\\
  \url{Peter.Bastian@iwr.uni-heidelberg.de}
}
\date{\today}

\begin{document}

\maketitle

\begin{abstract}
Die Heidelberger Numerikbibliothek wurde begleitend zu den Vorlesungen
\textit{Einführung in die Numerik} und \textit{Numerik} in der Programmiersprache C++
entwickelt und stellt
einfach zu benutzende Klassen für grundlegende Aufgaben in der Numerik bis hin zur Lösung
von gewöhnlichen Differentialgleichungen zur Verfügung. In fast allen Klassen ist der
benutzte Zahlentyp parametrisierbar so dass auch hochpräzise Rechnungen durchgeführt werden können.
\end{abstract}

\section{Einführung}

\subsection{Was ist HDNUM}

Die Heidelberger Numerikbibliothek (HDNUM) ist eine C++-basierte Bibliothek
zur Durchführung der praktischen Übungen zu den Vorlesungen \textit{Einführung in die
Numerik} und \textit{Numerik (gewöhnlicher Differentialgleichungen)}. Die aktuelle Version ist
unter
\begin{center}
\url{https://parcomp-git.iwr.uni-heidelberg.de/Teaching/hdnum}
\end{center}
verfügbar und wird mit dem Versionskontrollsystem \lstinline{git} verwaltet.
Spezifische Versionen können auf der jeweiligen Vorlesungswebseite veröffentlicht werden.

Ziele bei der Entwicklung von HDNUM waren i) die einfache Benutzbarkeit (inklusive einfacher 
Installation), 
ii) die Demonstration objektorientierter Programmierung in der Numerik sowie die
Möglichkeit zur Durchführung von Berechnungen mit beliebiger Genauigkeit auf Basis
der Gnu Multiple Precision\footnote{\url{https://gmplib.org}} Bibliothek.
Derzeit bietet HDNUM die folgende Funktionalität:
\begin{enumerate}[1)]
\item Klassen für Matrizen und Vektoren
\item Lösung linearer Gleichungssystem
\item Lösung nichtlinearer Gleichungssysteme
\item Lösung von Systemen gewöhnlicher Differentialgleichungen
\item Lösung der Poissongleichung mit Finiten Differenzen
\end{enumerate}

\subsection{Installation}

HDNUM ist eine \glqq{}header only\grqq{} Bibliothek und erfordert keine Installation
ausser dem Herunterladen der Dateien. Die aktuelle Version kann man mittels
\begin{lstlisting}[basicstyle=\ttfamily\footnotesize,language=bash,frame=single]
$ git clone https://parcomp-git.iwr.uni-heidelberg.de/Teaching/hdnum.git
\end{lstlisting}
herunterladen. Hierzu ist das Programm \lstinline{git} erforderlich, welches für alle
Betriebssysteme frei erhältlich ist. Alternativ wird üblicherweise auch ein 
komprimiertes \lstinline{tar}-archive auf der Homepage der jeweiligen Vorlesung angeboten.
Dies entpackt man mittels
\begin{lstlisting}[basicstyle=\ttfamily\small,frame=single]
$ tar zxvf hdnum-XX.tgz
\end{lstlisting}
In dem entpackten bzw. installiertem Verzeichnis findet man die folgenden Dateien und
Unterverzeichnisse:
\begin{itemize}
\item \lstinline{hdnum.hh}: Diese Header-Datei ist in ein C++-Programm einzubinden um HDNUM
nutzen zu können. 
\item Das Verzeichnis \lstinline{mystuff} ist für ihre Programme vorgesehen aber Sie können
natürlich jedes andere Verzeichnis nutzen. Wichtig ist nur, dass der Compiler die Datei 
\lstinline{hdnum.hh} findet. Im Verzeichnis \lstinline{mystuff} ist schon ein Beispielprogramm
um gleich loslegen zu können. Dieses Programm übersetzt man mit:
\begin{lstlisting}[basicstyle=\ttfamily\small,frame=single]
$ cd mystuff
$ g++ -I.. -o example example.cc
\end{lstlisting}
Diese Befehle setzen voraus, dass auf ihrem System der GNU C++-Compiler installiert ist.
Unter Windows oder für andere Compiler müssen Sie die Befehle entsprechend anpassen.
\item Das Verzeichnis \lstinline{examples} im HDNUM-Ordner enthält viele Beispiele
geordnet nach Programmierkurs, Numerik 0 und Numerik 1.
\item Das Verzeichnis \lstinline{src} im HDNUM-Ordner enthält den Quellcode
der HDNUM Bibliothek. Diese Dateien werden von \lstinline{hdnum.hh} eingebunden.
\item Das Verzeichnis \lstinline{programmierkurs} im HDNUM-Ordner enthält
die Folien zum Programmierkurs.
\item Das Verzeichnis \lstinline{tutorial} im HDNUM-Ordner enthält den Quellcode
für dieses Dokument.
\end{itemize}

\subsubsection*{GNU Multiple Precision Bibliothek}

HDNUM kann Berechnungen mit hoher Genauigkeit durchführen. Hierzu ist die GNU Multiple Precision
Bibliothek (GMP) erforderlich, 
welche Sie für viele Systeme kostenlos erhalten können. Um GMP nutzen zu können müssen 
Sie in der Datei \lstinline{hdnum.hh} die Zeile
\begin{lstlisting}[basicstyle=\ttfamily\small,frame=single]
#define HDNUM_HAS_GMP 1
\end{lstlisting}
\textit{auskommentieren}. Zusätzlich sind eventuell Compileroptionen notwendig damit der Compiler
die Headerdateien und Bibliotheken von GMP findet. Dies kann dann so aussehen:
\begin{lstlisting}[basicstyle=\ttfamily\footnotesize,frame=single]
$ g++ -I.. -I/opt/local/include -o example example.cc -L/opt/local/lib -lgmpxx -lgmp
\end{lstlisting}

\section{Lineare Algebra}

\subsection{Vektoren}

\subsection{Matrizen}

\subsection{LR Zerlegung}

\subsection{QR Zerlegung}

\begin{appendices}

\section{Kleiner Programmierkurs}

\section{Unix Kommandos}

\end{appendices}

\bibliographystyle{plain}
\bibliography{hdnum.bib}
 
\end{document}
